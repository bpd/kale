\chapter{Introduction}
\label{Intro}
\lhead{ \leftmark }

The Java Virtual Machine (JVM) has become ubiquitous in enterprise applications for its platform portability, performance, and stability.  Companies such as Google \cite{google-jvm}, Facebook \cite{facebook-hbase}, and Twitter \cite{java-at-twitter} all deploy the JVM at scale in their application stack.  The benefits of the JVM have enticed developers to build implementations of Python, Ruby, and over 100 domain specific languages that run on the JVM \cite{jvm-lang-summit}.  Languages like Jython and JRuby that target the JVM get access to generational garbage collection, native threading, and a highly optimized Just In Time (JIT) compiler.  

Regardless of the current level of adoption of other languages, the JVM was designed primarily to run Java.  All the concepts of Java -- classes, exceptions, strong typing, and the distinction between primitive and reference types -- are all represented in code targeting the JVM.  Therefore, language implementers are forced to express the concepts of their language in terms of JVM structures.  This becomes problematic when the language being implemented doesn't conform to JVM function invocation and typing semantics.  In order for languages using non-Java semantics to run on the JVM, significant workarounds must be employed.  These workarounds typically increase implementation complexity, cause code bloat, and decrease runtime performance.

However, JVM 7 includes a new feature, \texttt{invokedynamic}, that makes the JVM a more flexible and adaptable language host.  \texttt{invokedynamic} is a new bytecode instruction and runtime support library that together allow a lower level of access to class loading and linking operations within the JVM.  Language features like structural typing, which defines type equivalence based on types defining the same set of methods, can use this new link access to attain a level of performance that was not previously possible.

Details of the current linking and invocation semantics of the JVM are described in Chapter \ref{chapter:JVM}.  \texttt{invokedynamic} itself is described in detail in Chapter \ref{chapter:Invokedynamic}, and its integration into a structural type system is described in Chapter \ref{chapter:StructuralTyping}.  Finally, a compiler for a language that supports structural typing for the JVM through the use of \texttt{invokedynamic} is described in Chapter \ref{chapter:Kale}.


